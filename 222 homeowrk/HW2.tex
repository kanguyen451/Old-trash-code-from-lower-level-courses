\documentclass{article}
\usepackage{amsmath,amssymb,amsthm,latexsym,paralist}

\theoremstyle{definition}
\newtheorem{problem}{Problem}
\newtheorem*{solution}{Solution}
\newtheorem*{resources}{Resources}

\newcommand{\name}[2]{\noindent\textbf{Name: #1}\hfill \textbf{UIN: #2}}
\newcommand{\honor}{\noindent On my honor, as an Aggie, I have neither
  given nor received any unauthorized aid on any portion of the
  academic work included in this assignment. Furthermore, I have
  disclosed all resources (people, books, web sites, etc.) that have
  been used to prepare this homework. \\[2ex]
 \textbf{Electronic signature: \underline{ Kim Nguyen } }}
 
\newcommand{\checklist}{\noindent\textbf{Checklist:}
\begin{compactitem}[$\Box$] 
\item Did you type in your name and UIN? 
\item Did you disclose all resources that you have used? \\
(This includes all people, books, websites, etc.\ that you have consulted.)
\item Did you sign that you followed the Aggie Honor Code? 
\item Did you solve all problems? 
\item Did you submit  the .tex and .pdf files of your homework to the correct link on eCampus? 
\end{compactitem}
}

\newcommand{\problemset}[1]{\begin{center}\textbf{Problem Set #1}\end{center}}
\newcommand{\duedate}[1]{\begin{quote}\textbf{Due dates:} Electronic
    submission of \textsl{yourLastName-yourFirstName-hw2.tex} and 
    \textsl{yourLastName-yourFirstName-hw2.pdf} files of this homework is due on
    \textbf{#1} on \texttt{http://ecampus.tamu.edu}. You will see two separate links
    to turn in the .tex file and the .pdf file separately. Please do not archive or compress the files.
    \textbf{If any of the two files are missing, you will likely receive zero points for this 
    homework.}\end{quote} }

\newcommand{\N}{\mathbf{N}}
\newcommand{\R}{\mathbf{R}}
\newcommand{\Z}{\mathbf{Z}}


\begin{document}
\vspace*{-15mm}
\begin{center}
{\large
CSCE 222 [Section 501] Discrete Structures for Computing\\[.5ex]
Spring 2019 -- Hyunyoung Lee\\}
\end{center}
\problemset{2}
\duedate{Wednesday, 2/6/2018, before 10:00 p.m.}
\name{Kim Nguyen}{426007378}
\begin{resources} (All people, books, articles, web pages, etc.\ that
  have been consulted when producing your answers to this homework.)
\end{resources}
\honor

\bigskip

\noindent
\textbf{*** Please make sure that you are solving the correct problems from 
the \underline{8th Edition of the Rosen book}, not the 7th Edition! ***}

\medskip


%problem 1
\begin{problem} (5 points $\times$ 2 = 10 points) Section 1.4, Exercise 10 b) and c), page 57. 
\end{problem}
\begin{solution} .\\
b) \forall x ( C(x) \land D(x) \land F(x) ) \\
c) \exists x ( C(x) \land F(x) \land \neg D(x) )
\end{solution}

\begin{problem} (5 points $\times$ 2 = 10 points) Section 1.4, Exercise 20 c) and d), page 57.
\end{problem}
\begin{solution}.\\
c) P(-5) \land P(-3) \land P(-1) \land P(3) \land P(5)\\
d) P(1) \lor P(3) \lor P(5) 
\end{solution}

\begin{problem} (2.5 points $\times$ 4 = 10 points) Section 1.4, Exercise 36, page 58.
\end{problem}
\begin{solution}.\\
a) \forall x((-2 < x) \land (x < 3))\\
b) \forall x((0 $\leq$ x) \land (x < 5))\\
c) \exists x((-4 $\leq$ x) \land (x $\leq$ 1))\\
d) \exists x((-5 < x) \land (x < -1))\\
\end{solution}

\begin{problem} (5 points $\times$ 4 = 20 points) Section 1.5, Exercise 28 a), b), c) and d), page 71. 
\textsl{Justify your answer or give a counterexample.}
\end{problem}
\begin{solution}.\\
a) True\\
b) True\\
c) False\\
d) False\\
\end{solution}

\begin{problem} (10 points $\times$ 2 = 20 points) Section 1.6, Exercise 14 c) and d), page 83.
\end{problem}
\begin{solution}.\\
c) Conjunction\\
d) Hypothetical Syllogism
\end{solution}

\begin{problem} (5 points) Section 1.7, Exercise 2, page 95. 
\end{problem}
\begin{solution}.\\
\textbf{Proof:} Let integers n and m be even integers such that n = 2k and m = 2j for any integers j and k. Then m + n $\iff$ 2k + 2j $\iff$ 2(k + j) $\iff$ 2z. Since z is an integer then we can conclude that if m and n are even integers then the sum of m and n is also even.
\end{solution}

\begin{problem} (5 points) Section 1.7, Exercise 6, page 95.  
\end{problem}
\begin{solution}.\\
\textbf{Proof:} Let integers n and m be odd integers such that n = 2k + 1 and m = 2j + 1 for any integers j and k. Then mn $\iff$ (2k + 1)(2j + 1) $\iff$ 4kj + 2k + 2j + 1 $\iff$ 2(2kj + j + k) + 1 $\iff$ 2z + 1. Since z is an integer then we can conclude that if m and n are odd integers then the product of m and n is also odd. 
\end{solution}

\begin{problem} (10 points) Prove by \textit{contradiction} that 
if $n\ge 1$ is a perfect square, then $n+2$ is not a perfect square. 
\end{problem}
\begin{solution}.\\
Let p be the proposition that if integer n $\geq$ 1 is a perfect square, then n + 2 is not a perfect square. For proof by contradiction, we suppose that $\neg$p is true. Since $\neg$p states that integer n $\geq$ 1 is a perfect square and n + 2 is a perfect square. If $\neg$p is true then we can set n = $t^{2}$ and n + 2 = $s^{2}$ such that t and s are integers. Let n = 4, then n + 2 = 6 but 6 is not a perfect square. Thus, we have a contradiction.
\end{solution}

\begin{problem} (10 points) Prove by \textit{contradiction} that
at least three of any 25 days chosen must fall in the same month 
of the year.
\end{problem}
\begin{solution} .\\
Let p be the proposition that at least three of any 25 days chosen must fall in the same month of the year. For proof by contradiction, suppose that $\neg$p is true. Since $\neg$p states that there are less than three days of any 25 days chosen that must fall in the same month of the year. There are 12 months in a year and at most 2 must fall in the same month of the year. Then 12 $\times$ 2 is 24 but we must choose from 25 days so at least three days must be chosen in the same month of the year. Thus we have a contradiction.
\end{solution} 

\goodbreak
\checklist
\end{document}
